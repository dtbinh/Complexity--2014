%% Template Paper ISIT 2013
%%
%% October 2012, Stefan M. Moser
%% based on various earlier templates
%%
%% Please note that your paper must be no more than five pages in
%% the IEEEtran conference style as presented here (including figures,
%% references, etc.!)

\documentclass[conference,a4paper]{IEEEtran}

%% Conference papers do not typically use \thanks and this command
%% is locked out in conference mode. If really needed, such as for
%% the acknowledgment of grants, uncomment the following:
%\IEEEoverridecommandlockouts

\begin{document}

\sloppy

%% Paper Title
%% You can use linebreaks \\ within to get better formatting as
%% desired. 
\title{ISIT 2013 Paper Template\\Please Capitalize Important Words in Title} 


%% Author names and affiliations:
%%
%% Avoiding spaces at the end of the author lines is not a problem with
%% conference papers because we don't use \thanks or \IEEEmembership.
%%
%% For several authors with only one affiliation:
%%
% \author{
%   \IEEEauthorblockN{Hui-Ting Chang and Stefan M.~Moser}
%   \IEEEauthorblockA{Department of Electrical and Computer Engineering\\
%     National Chiao Tung University (NCTU)\\
%     Hsinchu, Taiwan\\
%     Email: \{email-of-hui-ting,email-of-stefan\}@ieee.org} 
% }
%%
%% For up to three affiliations:
%%
\author{
  \IEEEauthorblockN{Stefan M.~Moser}
  \IEEEauthorblockA{Dep. of Electrical \& Computer Eng.\\
    National Chiao Tung University\\
    Hsinchu, Taiwan\\
    Email: email-of-stefan@ieee.org} 
  \and
  \IEEEauthorblockN{Erdal Ar\i{}kan}
  \IEEEauthorblockA{Dep. of Electrical \& Electronics Eng.\\
    Bilkent University\\
    Ankara, Turkey\\
    Email: email\_erdal@university.edu}
  \and
  \IEEEauthorblockN{Elza Erkip}
  \IEEEauthorblockA{ECE Department\\ 
    Polytechnic Institute of New York University\\
    New York, USA\\
    Email: elza\_s-email-address@ieee.org}
}
%%
%% For over three affiliations, or if they all won't fit within the width
%% of the page, use this alternative format:
%%
% \author{
%   \IEEEauthorblockN{
%     Michael Shell\IEEEauthorrefmark{1},
%     Homer Simpson\IEEEauthorrefmark{2},
%     James Kirk\IEEEauthorrefmark{3}, 
%     Montgomery Scott\IEEEauthorrefmark{3} and
%     Eldon Tyrell\IEEEauthorrefmark{4}}
%   \IEEEauthorblockA{
%     \IEEEauthorrefmark{1}School of Electrical and Computer Engineering\\
%     Georgia Institute of Technology, Atlanta, Georgia 30332--0250\\ 
%     Email: see http://www.michaelshell.org/contact.html}
%   \IEEEauthorblockA{
%     \IEEEauthorrefmark{2}Twentieth Century Fox, Springfield, USA\\
%     Email: homer@thesimpsons.com}
%   \IEEEauthorblockA{
%     \IEEEauthorrefmark{3}Starfleet Academy, San Francisco, California 96678-2391\\
%     Telephone: (800) 555--1212, Fax: (888) 555--1212}
%   \IEEEauthorblockA{
%     \IEEEauthorrefmark{4}Tyrell Inc., 123 Replicant Street, Los Angeles, California 90210--4321}
% }


%% Use for special paper notices
%\IEEEspecialpapernotice{(Invited Paper)}


%% To balance the two columns, you should reduce the text-height of
%% the last page using the following command:
%%%%%%%%%%%%%%%%%%%%%%%%%%%%%%%%%%%%%%%%%%%%%%%%%%%%%%%%%%%%%%%%%%%%%
%\addtolength{\textheight}{-9.35cm}
%%%%%%%%%%%%%%%%%%%%%%%%%%%%%%%%%%%%%%%%%%%%%%%%%%%%%%%%%%%%%%%%%%%%%
%% with an appropriate value. This command must be place on the second
%% last page, i.e., for a one-page abstract here, for a two-page
%% abstract right after the \maketitle command.


%% Create the title:
\maketitle

%% Abstract: 
%% For the final version of the accepted paper, please make sure you
%% remove the comment "THIS PAPER IS ELIGIBLE FOR THE STUDENT PAPER
%% AWARD."
%%
\begin{abstract}
  This paper provides the instructions for the preparation of paper
  submissions and of final manuscripts for the Proceedings of ISIT
  2013.
\end{abstract}



\section{Introduction}

The 2013 IEEE International Symposium on Information Theory will be
held at ICEC Convention Center near Taksim Square at the heart of
Istanbul, Turkey, from Sunday, July 7 to Friday, July 12, 2013.


\section{Submission}

Paper submission is handled online using the EDAS system:
\begin{center}
  https://edas.info/newPaper.php?c=13285
\end{center}
The papers are restricted in length to \textbf{five pages} in the
IEEEtran-conference style as presented here (including figures,
references, etc.).

Each paper must be classified as ``Eligible for student paper award''
or ``Not eligible for student paper award''. Paper that are selected
to be eligible for the student paper award should also contain
\emph{``THIS PAPER IS ELIGIBLE FOR THE STUDENT PAPER AWARD''} as a
first line in the abstract of the submission. Note that this reference
must be removed again in the final manuscript!

The deadline for the submission is \textbf{January 27, 2013}.


\section{Proceedings}

Accepted papers will be published in full (up to five pages in
length). A hard-copy book of abstracts will also be distributed at the
Symposium to serve as a guide to the sessions.

The deadline for the submission of the final camera-ready paper is
\textbf{May 15, 2013}.  Accepted papers not submitted by that date
will not appear in the ISIT Proceedings and will not be included in
the technical program of the ISIT.


\section{Preparation of the Paper}

Only electronic submissions in form of a PDF file will be accepted.
The paper should be formatted in A4-format with sufficient margins at
the top and the bottom.  We highly encourage the authors to prepare
their papers with \LaTeX{} using the \LaTeX{} style file
\verb+IEEEtran.cls+, which can be easily obtained from
\begin{center}
  \small
  http://www.ctan.org/tex-archive/macros/latex/contrib/IEEEtran/
\end{center}
and using the \LaTeX{} template file \verb+isit_template.tex+ that
produced this document and that may be downloaded from the ISIT 2013
web site
\begin{center}
  http://www.isit2013.org/
\end{center}
Authors using other means to prepare their manuscripts should attempt
to duplicate the style of this example as closely as possible.

The style of references (e.g., \cite{shannon1948}), equations,
figures, tables, etc., should be the same as for the \emph{IEEE
  Transactions on Information Theory}. Example codes for figures and
tables can be found in the template file (they are commented out).

%% An example of a floating figure using the graphicx package.
%% Note that \label must occur AFTER (or within) \caption.
%% For figures, \caption should occur after the \includegraphics.
%%
% \begin{figure}[htbp]
%   \centering
%   \includegraphics[width=0.3\textwidth]{myfigure}
%   % where an .eps filename suffix will be assumed under latex,
%   % and a .pdf suffix will be assumed for pdflatex
%   \caption{Simulation results.}
%   \label{fig:sim}
% \end{figure}


%% An example of a double column floating figure using two subfigures.
%% (The subfigure.sty package must be loaded for this to work.)  The
%% subfigure \label commands are set within each subfigure command,
%% the \label for the overall fgure must come after \caption.  
%% \hfil must be used as a separator to get equal spacing
%%
% \begin{figure*}[htbp]
%   \centerline{\subfigure[Case I]{\includegraphics[width=2.5in]{subfigcase1}
%       % where an .eps filename suffix will be assumed under latex,
%       % and a .pdf suffix will be assumed for pdflatex
%       \label{fig:first_case}}
%     \hfil
%     \subfigure[Case II]{\includegraphics[width=2.5in]{subfigcase2}
%       % where an .eps filename suffix will be assumed under latex,
%       % and a .pdf suffix will be assumed for pdflatex
%       \label{fig:second_case}}}
%   \caption{Simulation results.}
%   \label{fig:sim}
% \end{figure*}


%% An example of a floating table. 
%% Note that, for IEEE style tables, the \caption command should come
%% BEFORE the table. Table text will default to \footnotesize as IEEE
%% normally uses this smaller font for tables.  The \label must come
%% after \caption as always.
%%
% \begin{table}[htbp]
%   % increase table row spacing, adjust to taste
%   \renewcommand{\arraystretch}{1.3}
%   \caption{An Example of a Table}
%   \label{tab:table_example}
%   \centering
%   % Some packages, such as MDW tools, offer better commands for making tables
%   % than the plain LaTeX2e tabular which is used here.
%   \begin{tabular}{|c||c|}
%     \hline
%     One & Two\\
%     \hline
%     Three & Four\\
%     \hline
%   \end{tabular}
% \end{table}


For proper typesetting of equations, the environment
\verb+\begin{IEEEeqnarray}+ is highly recommended. A short
introduction on how to use this tool can be found online at
\begin{center}
  http://moser.cm.nctu.edu.tw/manuals.html\#eqlatex
\end{center}

The affiliation shown for authors should constitute a sufficient
mailing address for persons who wish to write for more details about
the paper.


\section{Conclusion}

The conclusion goes here. 

For an appendix see comments in the template file.

%% Appendix:
%% If needed a single appendix is created by
%\appendix
%% If several appendices are needed, then the command
%\appendices
%% in combination with further \section-commands can be used.



%% Use \section* for acknowledgement
\section*{Acknowledgment}

The authors would like to thank various sponsors for supporting 
their research. 


%% References:
%% We recommend the usage of BibTeX:
%%
%\bibliographystyle{IEEEtran}
%\bibliography{definitions,bibliofile}
%%
%% where we here have assume the existence of the files
%% definitions.bib and bibliofile.bib.
%% BibTeX documentation can be obtained at:
%% http://www.ctan.org/tex-archive/biblio/bibtex/contrib/doc/
%%
%%
%%
%% Or manual references (pay attention to consistency!):
\begin{thebibliography}{1}
\bibitem{shannon1948}
  C.~E. Shannon, ``A mathematical theory of communication,''
  \emph{Bell System Techn. J.}, vol.~27, pp. 379--423 and 623--656,
  Jul. and Oct. 1948. 
\end{thebibliography}


\end{document}
