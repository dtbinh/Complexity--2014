\documentclass[12pt]{article}

\usepackage{amsmath}
\usepackage{xcolor}

\newcommand{\alert}[1]{{\color{red}#1}}

\title{Response to the Reviewers}
\author{Joshua Garland, Ryan James, and Elizabeth Bradley}
\date{\today}

\begin{document}

\maketitle

\section*{Response to the First Referee}

\emph{In this manuscript the authors present a way to explore the relationship
between predictability and complexity across a broad (but not complete) array of
forecast strategies such as random walk (last value), na\"ive (average value),
regression based (ARIMA) and a nonlinear method based on state space
reconstruction (LMA). They use weighted permutation entropy as a criterion/index
of predictive structure in the time series and illustrate their approach on
eight real-world datasets derived from three different systems (computer
performance traces).}

\emph{Overall the manuscript is well written and clearly understandable. The
analysis seems to be carried out carefully and the conclusions drawn are quite
reasonable. While I don't get the impression that this is a breakthrough paper I
can still recommend the manuscript for publication in PRE if the issues listed
below are addressed thoroughly.}

We thank the reviewer for their kind words and appreciate the time they put in
to reviewing our work.

\noindent\emph{Major issues:}

\emph{I don't get the reason for normalizing the three methods by the random
walk method (for example in Table I). As the authors write themselves this makes
the indicator vulnerable to a bias caused by the high influence of this
(arbitrarily selected) reference method. Why not just use a non-referential
estimator, which then would also allow to compare the inherent predictability of
different systems (as it is now the values obtained for each systems are
strongly influenced by the random walk estimate on that particular system and
thus can not be compared). This will also change the results of the fitting (the
authors explicitly exclude one value because of the effect caused by the
normalization). And it is not that the random walk predictor is the one standard
method that all other methods have to compete against.}

While we recognize the drawbacks inherent in using the industry-standard error
metric MASE, we chose it over other error metrics due to some of its favorable
qualities; namely that it is scale-free and comparable across different time
series. In our opinion this made it the superior candidate despite the bias it
can introduce.

\emph{In many parts the article reads almost like a review. There is a lot of
unnecessary redundancy (the entropy rate is quite low). In particular, the main
point (``The aim of the paper is not ... but ...'') is repeated far too many
times. The paper while easy to read could be shortened considerably without any
loss of information.}

Due to the broad selection of fields that this work draws upon, we thought it
necessary to provide background for each of them in case the reader is not
familiar with some subset of them. While this necessitated some redundancy,
\alert{we have made an effort to trim the fat and make the review portions more
streamlined.}

\noindent\emph{Minor issues:}

\emph{Page 1: Why mention the `halting problem'? I don't see a straightforward
connection to the problem at hand. The only connection is `undecidability' but
there are many other examples for that.}

Our intent in mentioning this theoretical result was to side-step the possible
response of simply ``always use the most sophisticated prediction scheme''. This
corollary implies that no matter how sophisticated and broad a scheme we
construct, it will not be the optimal one for some time series. Thus there is a
real need for determining when a better scheme exists. \alert{We have modified
the paragraph to make this point more clear.}

\emph{Page 10: `error between the predictions and the true continuations'
$\rightarrow$ `error of the predictions with respect to the true continuations'
Figure 6 on Page 11: It would be more intuitive to turn this into a `table of
figures' with the four methods as row labels on the left and the three systems
as column labels at the top. Also: Caption `for forecast of ... and all four
prediction strategies'.}

We have modified the text as the reviewer suggested, and agree that it reads
more clearly now. \alert{We have also modified the figure as suggested.}

\emph{First equation on Page 14: Shouldn't the index $j$ run from $i+1$ to
$i+\ell$ (instead of from $i$ to $i+\ell$)? You normalize by $\ell$ (not
$\ell+1$) values.}

\emph{Page 19: ``There are some exceptions: [...] Conversely, forecast errors
for the \texttt{col\_major} and \texttt{dgesdd$_5$} signals are higher than the
corresponding WPEs suggest --- except for the nonlinear LMA predictor.'' This is
not really an exception. As mentioned by the authors before, in general, the
result only holds under the assumption that a reasonably good model has been
used for the prediction. But for ill-matched datasets and models like in the
cases mentioned above this is just not the case so it is not a surprise that the
forecast errors are higher.}

\section*{Response to the Second Referee}

\emph{The problem of time series prediction is considered in the paper of J.
Garland et al. Namely, the authors suggest to quantify predictability in advance
(before using any concrete prediction scheme) based on the complexity estimate
(weighted permutation entropy, WPE). They hypothesize that WPE is related to the
best possible prediction error in a simple way. For several benchmark time
series (reflecting variations of a computer performance during execution of
different tasks), they obtain estimates of WPE and prediction error (the least
one over four techniques), and build a linear regression using seven or eight
resulting data points. They suppose that the obtained linear function can be
used to ``predict'' the best prediction error from the model-free WPE estimate.
Thus, the authors finally assume (as could be seen from the last section) that
WPE and optimal prediction error for all systems are related via the same
universal linear function with fixed coefficients. This is what I can understand
from the entire paper, though the problem is not so definitely described in the
Introduction.}

\emph{I think that the possibility to estimate the best prediction error in
advance from a relatively simple WPE characteristic would be interesting and
useful, if it could hold true for a wide class of systems. Such a question would
deserve a careful study as that started in the presented work. However, I have a
number of critical remarks and strongly doubt that the conjecture of the authors
can be valid for a reasonably wide class of systems (not even speaking of ``all
systems''). Other aspects of the paper are not so important, in my view, despite
some of them are fresh and interesting (e.g. taking a computer as an object of
prediction). My critical remarks are listed below.}

\emph{1) Introduction is too long, but the main problem is not formulated
definitely enough. I see that the main point is the approximately linear (or at
least one-to-one) relationship between WPE and the best prediction error. This
aim is not claimed explicitly in the Introduction. All other aspects of the
manuscript do not seem important enough to be published in a separate article.}

\emph{2) In particular, two primary findings mentioned on page 2 (Introduction)
are formulated quite vaguely. The first part of the first finding reads ``(i)
complexity of a noisy real-value time series is quantifiable by permutation
entropy''. However, it seems obvious from the previous research, including that
cited by the authors. The authors do not suggest PE or WPE here, they exploit
them as previously known approaches. It was previously known that PE is an
estimate of KS entropy as the authors also state. It was well known that KS
entropy relates to predictability and, in this sense, it is a measure of
complexity (see e.g. the work [G. Boffetta, M. Cencini, M. Falcioni, and A.
Vulpiani, ``Predictability: a way to characterize complexity'' // Physics
Reports 356 (2002) 367–474] which is not cited by the authors). Thus, the first
part of the first finding does not provide any new information. The second part
of the first finding reads ``(ii) complexity of the noisy real-valued time
series is correlated with prediction accuracy of an appropriate predictor''.
However, it seems also obvious and directly follows from the definition of KS-
entropy and its discussion in many papers including [Boffetta et al.] cited
above.}

\emph{3) By the way, the authors call ``fully complex'' the white noise process,
i.e. they equate predictability and complexity. However, they do not even
mention another approach to the notion of complexity which ascribes low
complexity to the white noise [the works e.g. by C.R. Shalizi and J. Crutchfield
carried out in the Santa Fe Institute from where one of the authors is]. It
would be appropriate to mention that ambiguity of the notion of complexity if
the latter plays an important role in many formulations of the authors.}

\emph{4) The second finding reads ``The way information is generated and
processed internally by a system plays a crucial role in the success of
different forecasting schema - and in the choice of which one is appropriate for
a given time series''. This statement also looks trivial. Indeed, an optimal
predictor for a linear stochastic system and low-dimensional nonlinear
deterministic system are quite different (an AR model versus a local or global
nonlinear model) due to different properties of the original systems. Thus,
there is nothing to be proven here. Probably, the authors imply their particular
benchmark signals (reflecting a computer behavior) but they do not claim that.
Thus, both primary findings seem to provide nothing new.}

\emph{5) The benchmark system - a computer as an object of prediction - seems to
be an unexpected and, probably, interesting choice but just as an additional
illustration of a statement which should be first shown for well-controlled and
well-understood paradigmatic systems (like low-dimensional maps or ordinary
differential equations). Indeed, it is not known in advance what error one
should expect for such a complicated object using a certain predictive
technique, what technique is optimal or close to optimal, etc. If the main point
of the research is the relationship between WPE and the best prediction error
(if any), selection of such a complicated object as computer performance
variations only introduces its own difficulties into the problem.}

\emph{6) The description of the choice of the benchmark system in the
Introduction is too long. On the other hand, if the computer signals are the
main interest of the work, then it should be claimed explicitly. Then, the paper
should be rewritten and submitted to a kind of engineering journals where it
might well be of interest.}

\emph{7) As for the choice of the four prediction techniques, some of them seem
superfluous. E.g. naive approach is just a global AR model of the zero order.
Random-walk is similar to the AR model of the first order where the coefficient
is not estimated by the least-squares but set equal to 1 (or this is an ARIMA
model with the first difference and ``zero ARMA part''). Inefficiencies of the
ARIMA model as compared to those choices in the paper can represent an
inappropriate procedure for the order selection rather than indicate
inappropriateness of the entire ARIMA approach. Thus, linear ARIMA and nonlinear
LMA might be sufficient (in combination with mathematical benchmark examples) to
represent close-to-optimal prediction errors.}

\emph{8) Conditions for inefficiency of simple prediction schemes (like naive
and random-walk) discussed in the paper look rather obvious as well and do not
give new information. Is it the purpose of the paper to discuss when and where a
concrete simple prediction scheme fails?}

\emph{9) A better justification of the prediction error metrics could be also in
order if this work is continued. At least, comparison of prediction errors
quantified with different metrics seems necessary. Especially, taking into
account the following major concerns.}

\noindent\emph{General remarks.}

\emph{I) Strict relationship between WPE and prediction error applicable to all
systems is hardly possible. In particular, there is a notion of epsilon-entropy
which tends to the KS entropy if epsilon (the cell size) tends to zero (see e.g.
[Boffetta et al.] cited above). For stochastic systems, the KS entropy is
infinite and epsilon-entropy varies strongly with the scale epsilon. Seemingly,
the permutation entropy estimate in this case will strongly depend on the choice
of the word length L. The authors concentrate on L = 6 and obtain coefficients
of the regression between WPE and prediction error only for that choice and for
their particular signals. However, other systems and other choices of L may
easily lead to other regression coefficients so that deciding about the best
possible prediction error from WPE seems impossible in general. It is good if
such a decision is possible for a certain narrow class of systems. However, the
authors do not even discuss to which class of systems their results apply.}

\emph{II) PE is invariant under an invertible nonlinear change of variables. The
prediction error measured as the authors suggest is not invariant. Thus, such a
simple change of variables can lead, at least, to a change in regression
coefficient between WPE and prediction error. Thus, the authors should also
discuss to which variable for that class of systems their result apply. I think
that the fact that the authors observe a certain regression line at all seems to
be due to small number of different signals they consider (only 7 or 8 different
data points, such a small number of points may lie approximately near a straight
line by chance).}

\emph{To summarize, the paper shows that the authors are well familiar with the
entire topic and highly qualified, they overview some well-known facts (WPE,
various prediction techniques, etc) and ``invent'' an unexpected test system
(computer performance data), they report about their attempts to play around
with all that staff. However, their results do not seem to provide any new
information to a specialist. Probably, this work can be continued to get some
more reliable results about concrete relationship between WPE and prediction
error for certain classes of systems (however, I doubt that it is possible to
derive any simple and widely applicable relationship like the linear function
suggested by the authors). Seemingly, it should then be a new paper which can be
hardly considered as a revised version of the present manuscript.}

\end{document}
