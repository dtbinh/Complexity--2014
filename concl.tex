\section{ Conclusions \& Future Work }\label{sec:conc}

No forecast strategy will be effective if a time series has no
predictive structure.  In practice, this almost-tautological statement
poses real problems; without knowing anything about the generating
process, it is difficult to determine how much predictive structure is
present in a given time series.  And even if predictive structure
exists, a given forecast method may not work, simply because it cannot
capture and/or exploit the structure that is present (e.g., a linear
model of a nonlinear process).  This too can be a serious practical
challenge.  If a forecast model is not producing good results, a
practitioner needs to know why: is the reason that the time series
contains no predictive structure---i.e., that no model will work---or
is the model that s/he is using simply not good enough?

In this paper, we have argued that redundancy~\cite{crutchfield2003}
is a useful definition of the inherent predictability of a time
series.  To operationalize that definition, we use an approximation of
the Kolmogorov-Sinai entropy~\cite{lind95}, estimated using a weighted
version of the permutation entropy of~\cite{bandt2002per}.  This WPE
technique---an ordinal calculation of forward information transfer in
a time series---is ideal for our purposes because it works with
real-valued data and is known to converge to the true entropy value.
Using a variety of forecast models and more than a hundred time-series
data sets from a computer-performance experiment, we have shown that
prediction accuracy is indeed correlated with weighted permutation
entropy: the higher the WPE, in general, the higher the prediction
error.  There are some exceptions: [[Liz to do: summarize \& discuss
    here.]]

% Further, the persistent WPE values of 0.5--0.6 for the {\tt
%   col\_major} trace are consistent with dynamical chaos, further
% corroborating the results of~\cite{mytkowicz09}.

An important practical corollary to this empirical correlation of
predictability and WPE is a practical strategy for assessing
appropriateness of forecast methods.  If the forecast produced by a
particular method is poor but the time series contains a significant
amount of predictive structure, one can reasonably conclude that that
method is inadequate to the task and that one should seek another
method.  The nonlinear LMA method, for instance, performs better in
most cases because it is more general.  The \naive ~method, which
simply predicts the mean, works better in noisy signals because it
effects a filtering operation.  The dashed line in
Figures~\ref{fig:wpe_vs_mase_best} and~\ref{fig:wpe_vs_mase_all} is a
useful empirical guideline for knowing when a model is not
well-matched to the task at hand: a $MASE$ score that is more than
{\color{red}[[what's the equation for that line, Ryan?]]} times the
WPE of the corresponding signal is an indication that that time series
has more predictive structure than that forecast model can capture and
exploit.

Given that information is a fundamental limitation in predictability,
then gathering and using more information is an obvious next step.
But there is an equally obvious tension here between data length and
prediction speed: a forecast that requires half a second to compute is
not useful for the purposes of real-time control of a computer system
with a MHz clock rate.  Another alternative is to sample several
system variables simultaneously and build multivariate models.  This
is a particular challenge in nonlinear LMA-type models, since
multivariate delay-coordinate embedding (e.g.,
\cite{cao-multivariate-embedding,deyle-sugihara2011}) can be
computationally prohibitive.  We are working on alternative methods
that sidestep that complexity.

%Paragraphs on the issues that come up, including one about the "amount
%of info" one: if one could sample more variables, for instance, one
%might be able to do a better job of predicting more-complex traces.
%Segue to some handwaving about multivariable LMA models; tie this back
%to the "computers are NLD systems" stuff in the intro.  This is a real
%challenge; current approaches to this modelling problem have the major
%issue of taking way too long to build.  And that's a big issue if
%you're trying not just to classify, but to predict.  In a system that
%runs at MHz speeds, a prediction that takes milliseconds to compute is
%not useful.
%\end{it}

Nonstationarity is a serious challenge in any time-series modeling
problem.  Indeed, one of the first uses of permutation entropy was
regime-shift detection for the purposes of recognizing seizure onset
in brainwave data~\cite{cao2004det}.  Recall that the signal in
Figure~\ref{fig:svd_ts_colored} was especially useful for the study in
this paper because it contained a number of different regimes.  We
segmented this signal visually, but one could imagine using
permutation entropy to do so instead.  Automating regime-shift
detection would be an important step towards a fully adaptive modeling
strategy, where old models are discarded and new ones are rebuilt
whenever shifts are detected.  WPE would be particularly powerful in
this scenario, as its value can suggest what kind of model might work
well in each new regime.

All of the experiments described in this paper involve a particular
physical system.  We chose that system because its behavior spans a
broad spectrum of classical time series patterns.  We believe that our
results will generalize far beyond this particular system; proving
that claim with examples is another important next step.  [[Liz to do:
    insert some examples.  Close with mention of what that could
    enable (viz., adaptive models like ones at end of prev paragraph
    used as the core of controllers).]]

% The way structure, information and complexity are processed by a
% generating process play a crucial role in the success of a given
% prediction scheme.
