\section{ Conclusions \& Future Work}\label{sec:conc}

\begin{enumerate}
\item  This need to be rewritten to conclude and address this paper...
\end{enumerate}



The results presented here suggest that permutation entropy---a ordinal
calculation of forward information transfer in a time series---is an effective
metric for predictability of computer performance traces. Experimentally, traces
with a persistent PE $\gtrapprox 0.97$ have a natural level of complexity that
may overshadow the inherent determinism in the system dynamics, whereas traces
with PE $\lessapprox 0.7$ seem to be highly predictable (viz., at least an order
of magnitude improvement in nRMSPE).Further, the persistent WPE values of 0.5--
0.6 for the {\tt col\_major} trace are consistent with dynamical chaos, further
corroborating the results of~\cite{mytkowicz09}.

If information is the limit, then gathering and using more information is an
obvious next step.  There is an equally obvious tension here between data length
and prediction speed: a forecast that requires half a second to compute is not
useful for the purposes of real-time control of a computer system with a MHz
clock rate.  Another alternative is to sample several system variables
simultaneously and build multivariate delay-coordinate embeddings.  Existing
approaches to that are computationally prohibitive
\cite{cao-multivariate-embedding}.  We are working on alternative
methods that sidestep that complexity.

%%This use of permutaiotn entropy is a useful application of information
%%theory to computer performance modeling because it gives you a simple
%%and fast way to decide whether building a model and trying to forecast
%%the future is worthwhile or if guessing the mean is just as effective.
%%End with a sentence tying back to power management and world peace.


%\begin{it}
%Summarize the results.

%Paragraphs on the issues that come up, including one about the "amount
%of info" one: if one could sample more variables, for instance, one
%might be able to do a better job of predicting more-complex traces.
%Segue to some handwaving about multivariable LMA models; tie this back
%to the "computers are NLD systems" stuff in the intro.  This is a real
%challenge; current approaches to this modelling problem have the major
%issue of taking way too long to build.  And that's a big issue if
%you're trying not just to classify, but to predict.  In a system that
%runs at MHz speeds, a prediction that takes milliseconds to compute is
%not useful.
%\end{it}

