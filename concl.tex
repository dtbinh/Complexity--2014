\section{ Conclusions \& Future Work }\label{sec:conc}

% To address the shortcomings mentioned in the previous section, we are
% in the process of repeating our own study with other metrics, other
% forecast methods, and time-series data from other systems.

Forecast strategies that are designed to capture predictive structure
are ineffective when signal complexity outweighs information
redundancy.  This poses a number of serious challenges in practice.
Without knowing anything about the generating process, it is difficult
to determine how much predictive structure is present in a noisy,
real-world time series.  And even if predictive structure exists, a
given forecast method may not work, simply because it cannot exploit
the structure that is present (e.g., a linear model of a nonlinear
process).  If a forecast model is not producing good results, a
practitioner needs to know why: is the reason that the data contain no
predictive structure---i.e., that no model will work---or is the model
that s/he is using simply not good enough?

In this paper, we have argued that redundancy~\cite{crutchfield2003}
is a useful definition of the inherent predictability of an empirical
time series.  To operationalize that definition, we use an
approximation of the Kolmogorov-Sinai entropy~\cite{lind95}, estimated
using a weighted version of the permutation entropy
of~\cite{bandt2002per}.  This WPE technique---an ordinal calculation
of forward information transfer in a time series---is ideal for our
purposes because it works with real-valued data and is known to
converge to the true entropy value. Using a variety of forecast models
and more than 150 time-series data sets from experiments and
simulations, we have shown that prediction accuracy is indeed
correlated with weighted permutation entropy: the higher the WPE, in
general, the higher the prediction error.  The relationship is roughly
logarithmic, which makes theoretical sense, given the nature of WPE,
predictability, and MASE.

An important practical corollary to this empirical correlation of
predictability and WPE is a practical strategy for assessing
appropriateness of forecast methods.  If the forecast produced by a
particular method is poor but the time series contains a significant
amount of predictive structure, one can reasonably conclude that that
method is inadequate to the task and that one should seek another
method.  The nonlinear LMA method, for instance, performs better in
most cases because it is more general.  (This is particularly apparent
in the \col and \svdfive examples.)
% , where the other two methods do not perform well.)
The \naive ~method, which simply predicts the mean, can work very well
on noisy signals because it effects a filtering operation.  The simple
random-walk strategy outperforms LMA, \arima, and the \naive
~method on the \gcc signal, which is extremely complex---i.e.,
extremely low redundancy.
%  \naive ~wins on \svdone with the exception to the 5 outliers.

The curves and shaded regions in Figures~\ref{fig:wpe_vs_mase_best}
and~\ref{fig:wpe_vs_mase_all} generalize and operationalize the
discussion in the previous paragraph.  These geometric features are a
preliminary, but potentially useful, empirical heurists for knowing
when a model is not well-matched to the task at hand: a point that is
below and/or to the right of the shaded regions on a plot like
Figure~\ref{fig:wpe_vs_mase_best} indicates the time series has more
predictive structure than the forecast model can capture and
exploit---and that one would be well advised to try another method.

These curves were determined empirically using a specific error metric
and a finite set of forecast methods and time series traces.  If one
uses a different error metric, the geometry---and the heuristic---will
no longer apply.  And while the methods and traces were chosen to be
representative of the practice, they are of course not completely
comprehensive.  It is certainly possible, for instance, that the
nonlinear dynamics of computer performance is subtly different from
the nonlinear dynamics of other systems.  Our preliminary results on
other systems (H\'enon, Lorenz, white noise, SFI ``A'', transformations
of the computer performance data) lead us to believe that our results
will generalize beyond the examples described in this paper.  We are
in the process of following up on that exploration with a broader
study of data, forecast methods, and error metrics.

{\color{blue}agreed, I don't think it addds anything}\alert{I'm not sure this paragraph adds anything.  Can/should we
  remove it?  Given that information is a fundamental limitation in
  predictability, then gathering and using more information is an
  obvious next step.  But there is an equally obvious tension here
  between data length and prediction speed: a forecast that requires
  half a second to compute is not useful for the purposes of real-time
  control of a computer system with a MHz clock rate, for instance.
  Another alternative is to sample several system variables
  simultaneously and build multivariate models.  This is a particular
  challenge in nonlinear LMA-type models, since multivariate
  delay-coordinate embedding (e.g.,
  \cite{cao-multivariate-embedding,deyle-sugihara2011}) can be
  computationally prohibitive.  We are working on alternative methods
  that sidestep that complexity.}

%Paragraphs on the issues that come up, including one about the "amount
%of info" one: if one could sample more variables, for instance, one
%might be able to do a better job of predicting more-complex traces.
%Segue to some handwaving about multivariable LMA models; tie this back
%to the "computers are NLD systems" stuff in the intro.  This is a real
%challenge; current approaches to this modelling problem have the major
%issue of taking way too long to build.  And that's a big issue if
%you're trying not just to classify, but to predict.  In a system that
%runs at MHz speeds, a prediction that takes milliseconds to compute is
%not useful.
%\end{it}

Nonstationarity is a serious challenge in any time-series modeling
problem.  [[Can destroy our heuristic: regime shifts can cause the
    test signal to look different from the training signal, which can
    skew the MASE score (and the points on the curve).  A RW MASE
    score larger than 1, however, as mentioned at end of prev section,
    can flag when this is happening.]]  Detecting regime shifts---and
adapting prediction models accordingly---is an important area of
future work.  Indeed, one of the first applications of permutation
entropy was to recognize the regime shift in brainwave data that
occurs when someone has a seizure~\cite{cao2004det}.  Recall that the
signal in Figure~\ref{fig:svd-ts-colored} was especially useful for
the study in this paper because it contained a number of different
regimes.  We segmented this signal visually, but one could imagine
using permutation entropy to do so instead.  Automating regime-shift
detection would be an important step towards a fully adaptive modeling
strategy, where old models are discarded and new ones are rebuilt
whenever the time series enters a new regime.  WPE would be
particularly powerful in this scenario, as its value can not only help
with regime-shift detection, but also suggest what kind of model might
work well in each new regime.  Of particular interest would be the
class of so-called \emph{hybrid systems}~\cite{hybrid}, which exhibit
discrete transitions between different continuous regimes---e.g., a
lathe that has an intermittent instability or traffic at an internet
router, whose characteristic normal traffic patterns shift radically
during an attack.  Effective modeling and prediction of these kinds of
systems is quite difficult; doing so adaptively and automatically---in
the manner that is alluded to at the end of the previous
paragraph---would be an even more important challenge.


