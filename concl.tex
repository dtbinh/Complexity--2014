\section{ Conclusions \& Future Work }\label{sec:conc}
{\color{red} NOT EDITABLE}
\begin{enumerate}
\item  This need to be rewritten to conclude and address this paper...
\end{enumerate}

Paragraph about PE as regime-shift detection.  Mention ``wpe has been
applied to predicting irregularities in brain wave data but no on one
has examined it's correlation with predictive structure. " here.

Moved the following material here from the results section, since
they're more general.  But there's a lot of repeated material that
will obviously need to get compressed and streamlined.

Just because there is forward information transfer, that does not mean
that an arbitrary predictor can interpret or utilize it, but WPE can
tell us when this structure is present.

These results support that time series with low to moderate complexity
($0\le WPE \le 0.85$) can be predicted more efficiently than a na\"ive
random walk \emph{and} that complexity can be qualitatively measured
for a real-valued noisy time series using WPE. This will allow
practitioners to stop spinning their wheels in the case of signals who
are simply better predicted with a simple strategy like random
walk. The analysis of these results illuminates an interesting point:
The way structure, information and complexity are processed by a
generating process plays a crucial role in the success of a given
prediction scheme.

As we discussed in Section \ref{sec:meaComplex} distinguishing
structured from unstructured complexity in the case of real-valued
time series is non trivial, i.e., distinguishing when a complex signal
omits predictable structure and when the signal is effectively random
is not a trivial task. As described in Section \ref{sec:intro}, for a
practitioner this can be frustrating because it can be nearly
impossible to find the source of faulty predictions: Is it simply that
we need to use a more ideal (possibly more advanced) predictor or is
it the case that the time series is simply so complex that using a
simple (yet inconsistent) forecast strategy such as random walk is the
best we can do. Fortunately, weighted permutation entropy (WPE) allows
us to make this distinction for noisy real-valued time series.

Needs to be more stuff in the WPE section for this to tie back to as
well as in the new related work section.  [[Joshua, can you elaborate
    on this so that Ryan can fill this in?]]

The results presented here suggest that permutation entropy---a ordinal
calculation of forward information transfer in a time series---is an effective
metric for predictability of computer performance traces. Experimentally, traces
with a persistent PE $\gtrapprox 0.97$ have a natural level of complexity that
may overshadow the inherent determinism in the system dynamics, whereas traces
with PE $\lessapprox 0.7$ seem to be highly predictable (viz., at least an order
of magnitude improvement in nRMSPE).Further, the persistent WPE values of 0.5--
0.6 for the {\tt col\_major} trace are consistent with dynamical chaos, further
corroborating the results of~\cite{mytkowicz09}.

If information is the limit, then gathering and using more information is an
obvious next step.  There is an equally obvious tension here between data length
and prediction speed: a forecast that requires half a second to compute is not
useful for the purposes of real-time control of a computer system with a MHz
clock rate.  Another alternative is to sample several system variables
simultaneously and build multivariate delay-coordinate embeddings.  Existing
approaches to that are computationally prohibitive
\cite{cao-multivariate-embedding}.  We are working on alternative
methods that sidestep that complexity.

%%This use of permutaiotn entropy is a useful application of information
%%theory to computer performance modeling because it gives you a simple
%%and fast way to decide whether building a model and trying to forecast
%%the future is worthwhile or if guessing the mean is just as effective.
%%End with a sentence tying back to power management and world peace.


%\begin{it}
%Summarize the results.

%Paragraphs on the issues that come up, including one about the "amount
%of info" one: if one could sample more variables, for instance, one
%might be able to do a better job of predicting more-complex traces.
%Segue to some handwaving about multivariable LMA models; tie this back
%to the "computers are NLD systems" stuff in the intro.  This is a real
%challenge; current approaches to this modelling problem have the major
%issue of taking way too long to build.  And that's a big issue if
%you're trying not just to classify, but to predict.  In a system that
%runs at MHz speeds, a prediction that takes milliseconds to compute is
%not useful.
%\end{it}

