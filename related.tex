\section{Related Work }\label{sec:related}

Modeling time-series data for the purposes of prediction is an old
problem that dates back at least to Yule's 1927 invention of
autoregression~\cite{Yule27}.  Since then hundreds, if not thousands,
of prediction strategies have been developed for a wide variety of
prediction tasks.  The purpose of this paper is not to add a new
method to this arsenal, nor to assess or compare the effectiveness of
these existing methods.  Our goals are broader: (i) to empirically
quantify the predictive structure that is present in a real-valued
scalar time series and (ii) to explore how that inherent complexity
affects the performance of those existing prediction methods.  It
would, of course, be neither practical nor interesting to report
results for every existing forecast method; instead, we choose a
representative set, which are described in Section~\ref{sec:model}.

Quantifying predictability, which is sometimes called ``predicting
predictability,'' is also not a new problem.  Most solutions fall into
two categories: model-based error analysis and model-free information
analysis.
%
%Actually this first class are both just analyzing model error distributions but some are local and some are global models but both are really just error
%
%predicting local predictive capacity (radial basis functions stuff, trying to predict error bounds on next forecast based on ensemble uncertainty) but does not aggregate tell you at what level the time series exhibits complexity only locally predictive structure, this actualy gets at the interesting point that different regions of a time series may exhibit differnt levels of complexity which we will illustrate with \svd
%
The first class focuses on errors produced by a fixed forecasting
schema.  This can proceed locally or globally.  The local version of
this type of analysis approximates error distributions for different
regions of a time-series model using local ensemble in-sample
forecasting.  [[Joshua, we probably want a quick definition of
    in-sample and out-of-sample.]] These distributions are then used
to estimate out-of-sample forecast errors [[in that area of]] the time
series model.  For example, Smith {\sl et al.} make in-sample
forecasts using ensembles around selected points in the time series in
order to predict the local predictability of a time series
reconstructed using delay-coordinate embedding~\cite{Smith199250}.
This approach can be used to show that different portions of a time
series can exhibit varying levels of local predictive uncertainty.  We
expand on this finding later in this paper with a time series that
exhibits interesting regime shifts.

Local model-based error analysis works quite well, but it only
approximates the \emph{local} predictive uncertainty \emph{in relation
  to a fixed model}.  It cannot quantify the inherent redundancy of a
time series~\cite{crutchfield2003} and thus cannot be used to draw
conclusions about predictive structure that may be usable by other
forecast methods.
%
%Distrubtion of error. For many methods, if your error is not normally distributed this signals that there is stil more predictve structure to that could be used by for example a larger order ARMA process. But if error is normally distributed, this suggests that you have used up all the predictive structure that **that** model can use, this doesn't quantify if preidctuce structure exists that isn't being used by this process. For example, nonlinear structure which is ignored by a linear predictor.
%
Global model-based error analysis moves in this direction.  It uses
out-of-sample error distributions, computed {\sl post facto} from a
class of models, to determine which of those models was best.  When
building an autoregressive model, for example, it is common to verify
that the model error is normally distributed.  If it is not, that
indicates that there is linear structure in the time series that the
model-building process failed to use.  The problem with this approach
is that if the error is normally distributed it can not be said that
the practitioner is using the ideal forecasting schema---e.g., if
nonlinear structure is present that linear modeling is oblivious
to---but only if this class of model used all the structure it was
able to process.

A practice known as deterministic vs. stochastic
modeling~\cite{Casdagli92dvsplots, weigend-book} bridges the gap
between local and global error analysis. The basic idea is to
construct a series of local linear fits (single point) to global
linear fits (fitting all points known at that time), increasing the
number of points in the fit slowly. Then analyze how the average
out-of-sample error changes as a function of number of points in the
linear fits. The shape of such a graph indicates how much determinism
vs. stochasticity is present in a time series, see
\cite{Casdagli92dvsplots, weigend-book} for more information.

The model-based error analysis methods described in the previous three
paragraphs are based on specific assumptions about the underlying
generating process and knowledge about what will happen to the error
if those assumptions hold or fail.  Model-free information analysis
moves away from those restrictions.  Our approach falls into this
class.  In particular, we empirically measure the inherent complexity
of a time series using an approximation of the Kolmogorov-Sinai
entropy, then correlate that complexity with the predictive accuracy
of forecasts made a number of different schema for that time series.

[[I think we need a short paragraph about redundancy in here
    somewhere.  We keep referring to it as if the reader knows it...]]

Using entropy and redundancy to quantify the inherent predictability
of a time series is not a new idea.  Past methods for this, however,
(e.g.,~\cite{Shannon1951, mantegna1994linguistic}) have hinged on
knowledge of the \emph{generating partition} of the generating
process, which partitions observations into symbols in a way that
preserves the underlying dynamics~\cite{lind95}.  This is a luxury
that is rarely, if ever, afforded to an analyst, since one needs to
know the underlying dynamics in order to construct a generating
partition.  Even if the dynamics are known, these partitions are
difficult to compute and often have fractal
boundaries~\cite{eisele1999}.

Unfortunately, using any partition (such as those produced by simply
binning the data) other than the generation partition induces spurious
complexity to the resulting symbolic sequence, thus misrepresenting
the entropy of the underlying system~\cite{bollt2001}.

It seems desirable to quantify how much information prior observations of a system lend to future values, the \emph{mututal information} between, say, the past $n$ observations at the current one. This quantity is known as the \emph{redundancy}. For example, this would be the amount of information that delay coordinate embedding captures. For IID random processes,  the redundancy is zero, since all observations are independent of each other. On the other hand, any deterministic system --- including chaotic ones --- can be perfectly predicted if observed for long enough~\cite{weigend-book}. Unfortunately it is quite difficult to estimate the redundancy in practice for an arbitrary, real-valued time series. Doing so requires knowing either the Kolmolgorov-Sinai entropy or the \emph{full} positive Lyapunov spectrum. As discussed above, computing the entropy is quite difficult. Similarly, computing the positive Lyapunov exponent(s) from an arbitrary time series is also fraught with difficulties, and is effectively impossible if there is significant noise or the generating system is stochastic.

Our approach expands on \cite{bandt2002per} to approximate normalized redundancy of a time series as a proxy for predictability. This method is different from the current literature because it does not rely on generating partitions, and as such does not suffer from the previously mentioned biases. In addition no knowledge of the underlying generating process is necessary, e.g, linear, nonlinear, the Lyapunov spectrum, etc. In fact, our approach uses no assumption about the underlying generating process, does not require generating partitions, and is applicable to noisy real-valued time series from all classes of systems.


{\color{red}
Maybe talk about ``
wpe has been applied to predicting irregularities in brain wave data but no on one has examined it's correlation with predictive structure. "
but kind of puts the cart before the horse

[[ryan]] i tried writing a paragraph about this around 7 times, and couldn't get anything that sounded reasonable.
}

