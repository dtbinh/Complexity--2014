\section{Related Work }\label{sec:related}

Modeling time-series data for the purposes of prediction dates back at
least to Yule's 1927 invention of autoregression~\cite{Yule27}.  Since
then hundreds, if not thousands, of strategies have been developed for
a wide variety of prediction tasks.  The purpose of this paper is not
to add a new weapon to this arsenal, nor to assess or compare the
effectiveness of existing methods.  Our goals are more general: {\sl
  (i)} to empirically quantify the predictive structure that is
present in a real-valued scalar time series and {\sl (ii)} to explore
how the performance of prediction methods is related to that inherent
complexity.  It would, of course, be neither practical nor interesting
to report results for every existing forecast method; instead, we
choose a representative set, as described in Section~\ref{sec:model}.

Quantifying predictability, which is sometimes called ``predicting
predictability,'' is not a new problem.  Most solutions to it fall
into two categories: model-based error analysis and model-free
information analysis.
%
%Actually this first class are both just analyzing model error distributions but some are local and some are global models but both are really just error
%
%predicting local predictive capacity (radial basis functions stuff, trying to predict error bounds on next forecast based on ensemble uncertainty) but does not aggregate tell you at what level the time series exhibits complexity only locally predictive structure, this actualy gets at the interesting point that different regions of a time series may exhibit differnt levels of complexity which we will illustrate with \svd
%
The first class focuses on errors produced by a fixed forecasting
schema.  This analysis can proceed locally or globally.  The local
version approximates error distributions for different regions of a
time-series model using local ensemble in-sample
forecasting\footnote{The terms ``in sample'' and ``out of sample'' are
  used in different ways in the forecasting community.  The meanings
  that take here are distinguished by the part of the time series that
  is the focus of the prediction: the observed data for the former and
  the unknown future for the latter.  In-sample
  forecasts---comparisons of predictions generated from \emph{part} of
  the observed time series---are useful for assessing model error and
  prediction horizons, among other things.}.
%
These distributions are then used as estimates of out-of-sample
forecast errors in those regions.  For example, Smith {\sl et al.}
make in-sample forecasts using ensembles around selected points in
order to predict the local predictability of that time
series~\cite{Smith199250}.  This approach can be used to show that
different portions of a time series can exhibit varying levels of
local predictive uncertainty.  We expand on this finding later in this
paper with a time series that exhibits interesting regime shifts.

Local model-based error analysis works quite well, but it only
approximates the \emph{local} predictive uncertainty \emph{in relation
  to a fixed model}.  It cannot quantify the inherent predictability
of a time series and thus cannot be used to draw conclusions about
predictive structure that may be usable by other forecast methods.
%
%Distrubtion of error. For many methods, if your error is not normally distributed this signals that there is stil more predictve structure to that could be used by for example a larger order ARMA process. But if error is normally distributed, this suggests that you have used up all the predictive structure that **that** model can use, this doesn't quantify if preidctuce structure exists that isn't being used by this process. For example, nonlinear structure which is ignored by a linear predictor.
%
Global model-based error analysis moves in this direction.  It uses
out-of-sample error distributions, computed \emph{post facto} from a
class of models, to determine which of those models was best.  After
building an autoregressive model, for example, it is common to
calculate the errors and verify that they are normally distributed.
If they are not, that indicates that there is structure in the time
series that the model-building process was unable to capture and use.
The problem with this approach is lack of generality.  
\label{page:normal-errors}
Normally distributed errors indicate that a model has captured the
structure in the data insofar as is possible, \emph{given the
  formulation of that particular model} (viz., the best possible
linear fit to a nonlinear dataset).  This gives no indication as to
whether another modeling strategy might do better.


A practice known as deterministic/sto\-chas\-tic
modeling~\cite{Casdagli92dvsplots, weigend-book} bridges the gap
between local and global approaches to model-based error analysis.
The basic idea is to construct a series of local linear fits,
beginning with a few points and working up to to a global linear fit
that includes all known points, and then analyze how the average
out-of-sample error changes as a function of number of points in the
fit. The shape of such a graph indicates the amounts of determinism
and stochasticity are present in a time series.

The model-based error analysis methods described in the previous three
paragraphs are based on specific assumptions about the underlying
generating process and knowledge about what will happen to the error
if those assumptions hold or fail.  Model-free information analysis
moves away from those restrictions.  Our approach falls into this
class.  Our goal is to empirically measure the inherent complexity of
a time series, then correlate that complexity with the predictive
accuracy of forecasts made using a number of different methods.  

We build on the notion of \emph{redundancy} that was introduced on
page~\pageref{page:redundancy}, which formally quantifies how
information propagates forward through a time series:
% how much information prior observations of a system lend to future
% values
i.e., the mutual information between the past $n$ observations and the
current one.
% took this out because DCE hasn't been introduced yet: 
% 
% For example,
% this would be the amount of information that delay coordinate
% embedding captures.
The redundancy of i.i.d. random processes is zero, since all
observations are independent of one another.  On the other hand, any
deterministic system---including chaotic ones---has maximal redundancy
and thus can be perfectly predicted if observed for long
enough~\cite{weigend-book}.  In practice, it is quite difficult to
estimate the redundancy of an arbitrary, real-valued time series.
Doing so requires knowing either the Kolmolgorov-Sinai entropy or the
values of all positive Lyapunov exponents of the system.  Both of
these calculations are difficult---the latter particularly so if the
data are very noisy or the generating system is stochastic.

Using entropy and redundancy to quantify the inherent predictability
of a time series is not a new idea.  Past methods for this, however,
(e.g.,~\cite{Shannon1951, mantegna1994linguistic}) have hinged on
knowledge of the \emph{generating partition} of the underlying
process, which lets one transform real-valued observations into
symbols in a way that preserves the underlying dynamics~\cite{lind95}.
Different projections---e.g., simply binning the data---can create
spurious complexity in the resulting symbolic sequence and thus
misrepresent the entropy of the underlying system~\cite{bollt2001}.
Generating partitions are luxuries that are rarely, if ever, afforded
to an analyst, since one needs to know the underlying dynamics in
order to construct one.  And even if the dynamics are known, these
partitions are difficult to compute and often have fractal
boundaries~\cite{eisele1999}.

We sidestep these issues by using a variant of the \emph{permutation
  entropy} of Bandt and Pompe~\cite{bandt2002per} to estimate a value
for the Kolmogorov-Sinai entropy of a real-valued time series---and
thus the redundancy in that data, which is an effective proxy for
predictability.  This differs from existing approaches in a number of
ways.  It does not rely on generating partitions---and thus does not
introduce bias into the results if one does not know the dynamics or
cannot compute the partition.  It makes no assumptions about, and
requires no knowledge of, the underlying generating process: linear,
nonlinear, the Lyapunov spectrum, etc.  These features make our
approach applicable to noisy real-valued time series from all classes
of systems.

